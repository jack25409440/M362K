\documentclass[12pt]{article}

\usepackage{graphicx}
\usepackage[margin=1.0in]{geometry}
\usepackage{amsmath}
\usepackage{cases}
\usepackage{amsfonts}
\usepackage{amssymb}
\usepackage{grffile}
\usepackage{setspace}

\setlength\parindent{0pt}

\author{Xiaohui Chen \\EID: xc2388}
\title{M 362K Pre-Class Work for 1/22}

\begin{document}
\maketitle
\begin{spacing}{2.0}

\section*{1-46}
\subsection*{(a)}
This problems can be considered as inserting 2 borders into 6 slots. However, since each bin can have 0 balls, the total number of distribution is $6^2=36$

\subsection*{(b)}
This problem has the same setting, but each bin should have at leat 1 ball. Therefore the total number of distribution should be ${}_{6}C_{2}= 15$


\section*{1-53}
\subsection*{(a)}
$(x+3)^4= {}_{4}C_{0} x^4+ {}_{4}C_{1} x^3y^1+ {}_{4}C_{2}x^2y^2+ {}_{4}C_{3}x^1y^3 + {}_{4}C_{4}y^4= x^4+ 4x^3y^1 + 6x^2y^2 +4x^1y^3+ y^4$

\section*{Sample Exam 8}
Hamburger, French Fries, Coke\\
Hamburger, French Fries, Diet Coke\\
Hamburger, French Fries, Chocolate Shake\\
Hamburger, Potato Chips, Coke\\
Hamburger, Potato Chips, Diet Coke\\
Hamburger, Potato Chips, Chocolate Shake

Cheeseburger, French Fries, Coke\\
Cheeseburger, French Fries, Diet Coke\\
Cheeseburger, French Fries, Chocolate Shake\\
Cheeseburger, Potato Chips, Coke\\
Cheeseburger, Potato Chips, Diet Coke\\
Cheeseburger, Potato Chips, Chocolate Shake

Double-Double, French Fries, Coke\\
Double-Double, French Fries, Diet Coke\\
Double-Double, French Fries, Chocolate Shake\\
Double-Double, Potato Chips, Coke\\
Double-Double, Potato Chips, Diet Coke\\
Double-Double, Potato Chips, Chocolate Shake



\end{spacing}
\end{document} 