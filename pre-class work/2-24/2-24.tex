\documentclass[12pt]{article}

\usepackage{graphicx}
\usepackage[margin=1.0in]{geometry}
\usepackage{amsmath}
\usepackage{cases}
\usepackage{amsfonts}
\usepackage{amssymb}
\usepackage{grffile}
\usepackage{setspace}

\setlength\parindent{0pt}

\author{Xiaohui Chen \\EID: xc2388}
\title{M 362K Pre-Class Work for 2/26}

\begin{document}
\maketitle
\begin{spacing}{2.0}

\section*{4-3}

For option (b), the total number of grains of rice is $2+2^2+2^4+ \cdot+ 2^63 = =2^{64}-1 > 1000000000 $

Therefore option (b) is better

\section*{4-6}

\subsection*{(a)}

The probability for each outcome is equal, which is $\frac{1}{6}$

$\therefore Pr(X \ge 5)= Pr(X=5)+Pr(X=6)= \frac{1}{3}$

\subsection*{(b)}

$mean=E[X]=\frac{1+6}{2}=3.5$

\subsection*{(c)}

$median=\frac{3+4}{2}=3.5$

\subsection*{(d)}

$Var[X]=\frac{6^2-1}{12}=\frac{35}{12}$

$\sigma=\sqrt{Var[X]}= \sqrt{\frac{35}{12}}= 1.70783$

\section*{4-14}

Let n denote the total number of throws. Let p denote probability of success

$E[M]=n*p= 12*0.8= 9.6$

$Var[M]= n*p*(1-p)= 12*0.8*0.2=1.92$

$Pr(M \le 10)= 1- Pr(M> 10) =  Pr(M=11)+ Pr(M=12)= 1- {}_{12}C_{11} 0.8^{11}*0.2^{1} + 0.8^{12}  =0.7251 $

\section*{4-17}

$Pr(X=1)= {}_{3}C_{1} 0.9*0.1^2 = 0.027$

$Pr(X\ge2)= Pr(X=2)+ Pr(X=3)={}_{3}C_{2} 0.9^2*0.1 + 0.9^3 = 0.972$

$E[X]= n*p= 3*0.9=2.7$

$Var[X]= n*p*(1-p)= 3*0.9*0.1= 0.27$


\end{spacing}
\end{document} 