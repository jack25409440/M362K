\documentclass[12pt]{article}

\usepackage{graphicx}
\usepackage[margin=1.0in]{geometry}
\usepackage{amsmath}
\usepackage{cases}
\usepackage{amsfonts}
\usepackage{amssymb}
\usepackage{grffile}
\usepackage{setspace}

\setlength\parindent{0pt}

\author{Xiaohui Chen \\EID: xc2388}
\title{M 362K Post-Class Homework 1}

\begin{document}
\maketitle
\begin{spacing}{2.0}

\section*{1-7}
For each of the digit positions, there are 10 possible outcomes

For the first and third alpha positions, each has $26-3=23$ possible outcomes since I,O and Q are excluded

For the second alpha position, there are 26 outcomes

Let $\#plates$ denotes the number of possible licence plates, then $\#plates= 10^{3}* 23^{2} * 26= 1.375*10^{7}$

\section*{1-9}
From the question we can know that the first digit has $10-2=8$ outcomes and the fourth digit has $10-1=9$ outcomes

Let $\#phone$ denotes the possible number of ten-digit phone numbers. Therefore,

$$\#phone=8*9*10^{10-2}=7.2*10^{9}$$

\section*{1-12}
According to the question, the number of possible outcomes for the trifeeta with 8 horses is ${}_{8}P_{1} *{}_{7}P_{1} * {}_{6}P_{1}$

Therefore, $Pr(winning\ ticket)=\frac{1}{{}_{8}P_{1} *{}_{7}P_{1} * {}_{6}P_{1}}=\frac{1}{8*7*6}= \frac{1}{336}$

\section*{1-17}
According to the question, there are 11 possible floors and 7 people. The total number of possible outcomes is $11^{7}$

If no two will get of at the same floor, this is equivalent to the permutation of 11 floors chosen 7, which is ${}_{11}P_{7}$

$\therefore Pr(no\ two\ get\ of\ at\ the\ same\ floor)= \frac{{}_{11}P_{7}}{11^7}= \frac{11!}{(11-7)! 11^7}\approx 0.085$

\section*{1-24}
\subsection*{(a)}
There are 13 toppings, 2 sizes and 2 types of crust

Let $\#2-top-pizza$ be the number of different two-topping pizzas

$\therefore \#2-top-pizza= {}_{13}C_{2}* {}_{2}C_{1}* {}_{2}C_{1}= 312$

\subsection*{(b)}
There are 6 meat toppings, 7 vegetable toppings, 2 sizes and 2 types of crust

Let $\#2-top-mv-pizza$ be the number of different two-topping pizzas with exactly one meat and exactly one vegetable topping

$\therefore \#2-top-mv-pizza= {}_{6}C_{1}*{}_{7}C_{1}* {}_{2}C_{1}* {}_{2}C_{1}=168$

\subsection*{(c)}
There are 7 vegetable toppings, 2 sizes and 2 types of crust

Let $\#4-top-v-pizza$ be the number of different four-topping vegetarian pizzas

$\therefore \#4-top-v-pizza= {}_{7}C_{4}* {}_{2}C_{1}* {}_{2}C_{1}=140$

\section*{1-32}
The number of possible words with 4 letters is indeed the permutation of the 4 letters. However, since the order of letter O does not matter, the number of possible words is therefore $\frac{4!}{2}= 12$

\section*{1-36}
\subsection*{(a)}
This is equivalent to dividing 25 girls into 3 groups of 5,5,15 girls respectively

Let $\#partition$ denotes the number of such partition

$\therefore \#partition= \left(
\begin{array}{ccc}
  & 25 & \\
5 & 5 & 15 
\end{array}
\right)= \frac{25!}{5!*5!*15!}=823727520$

\subsection*{(b)}
This question is also equivalent to 25 girls into 3 groups of 5,5,15 girls respectively (2 teams of 5 people and 1 group of 15 not playing)

Let $\#partition$ denotes the number of such partition

$\therefore \#partition= \left(
\begin{array}{ccc}
  & 25 & \\
5 & 5 & 15
\end{array}
\right)= \frac{25!}{5!*5!*15!}=823727520$

\end{spacing}
\end{document}