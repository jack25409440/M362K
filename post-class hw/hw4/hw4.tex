\documentclass[12pt]{article}

\usepackage{graphicx}
\usepackage[margin=1.0in]{geometry}
\usepackage{amsmath}
\usepackage{cases}
\usepackage{amsfonts}
\usepackage{amssymb}
\usepackage{grffile}
\usepackage{setspace}

\setlength\parindent{0pt}

\author{Xiaohui Chen \\EID: xc2388}
\title{M 362K Post-Class Homework 4}


\begin{document}
\maketitle
\begin{spacing}{2.0}

\section*{2-27}

\subsection*{(a)}
We know that $Pr(car)=\frac{1}{3}$ and $Pr(goat)=\frac{2}{3}$

If you choose to keep the original door, that means you have to select the correct door at the beginning

$\therefore Pr(car|without\ changing\ doors)=Pr(car)= \frac{1}{3}$

\subsection*{(b)}

If I select the door with goat in the first place, then I will win a car for sure if I change the door after the revelation. Otherwise I will be certain that I will win a goat

$\therefore Pr(car|changing\ doors)=\frac{2}{3}*1+ \frac{1}{3}*0=\frac{2}{3}$

\section*{2-29}
From the question, we know that $Pr(smoker|circulation)= 2x$, $Pr(smoker|non-circulation)=x$, $Pr(circulation)=0.25$

$\therefore Pr(circulation|smoker)= \frac{Pr(smoker|circulation) * Pr(cirulation)}{Pr(smoker|circulation) * Pr(cirulation) + Pr(smoker|non-circulation)*Pr(non-curculation)}= \frac{2x*0.25}{2x*0.25+0.75x}=\frac{2}{5}$

Therefore the answer is (C)

\section*{2-41}
\begin{tabular}{|c|c|c|c|}
  \hline
  % after \\: \hline or \cline{col1-col2} \cline{col3-col4} ...
    & A & A' &   \\
  \hline
  B & a*b & (1-a)*b & b \\
  \hline
  B' & a*(1-b) & (1-a)*(1-b) & 1-b \\
  \hline
    & a & 1-a &  1 \\
  \hline
\end{tabular}

\subsection*{(a)}
According to the Venn box diagram shown above, $Pr(A\cap B')=a*(1-b) =Pr(A)*Pr(B')$

Therefore, A and B' are independent

\subsection*{(b)}
$Pr(A'\cap B')=(1-a)*(1-b)= Pr(A')*Pr(B')$

Therefore, A' and B' are independent

\subsection*{(c)}
$Pr(A' \cap B)=(1-a)*b= Pr(A')*Pr(B)$

Therefore, A' and B are independent

\section*{2-44}
$Pr(A\cap B')=Pr(A)*Pr(B')=Pr(A)*(1-Pr(B))=Pr(A)-Pr(A)*Pr(B)$

Similarly, we get $Pr(A'\cap B)=Pr(B)-Pr(A)*Pr(B)$

We set $Pr(A\cap B)=Pr(A)*Pr(B)=x$

$\therefore (0.2+x)*(0.3+x)=x$

$x=0.2$ or $x=0.3$

$P(A)=0.2+0.2=0.4$ or $0.2+0.3=0.5$

$P(B)=0.3+0.2=0.5$ or $0.3+0.3=0.6$

$\therefore Pr(A\cup B)=Pr(A)+Pr(B)-Pr(A)*Pr(B)=0.4+0.5-0.2= 0.7$ or $0.5+0.6-0.3=0.8$

\section*{2-52}
$Pr(first-face)=\frac{12}{52}$ and $Pr(second-face|first-face)=\frac{11}{51}$

$\therefore Pr(first-face\cap second-face)= Pr(first-face)* Pr(second-face|first-face)= \frac{12}{52}* \frac{11}{51}= \frac{11}{221}\approx 0.05$

\section*{2-55}
Let the number of blue balls be x

$Pr(same\ color)=Pr(blue \cap blue)+ Pr(red\cap red)= \frac{4}{10}*\frac{16}{16+x}+\frac{6}{10}*\frac{x}{16+x}=0.44$

$\therefore x=4$

The number of balls in the second urn is 20. The answer is (B)

\end{spacing}
\end{document} 