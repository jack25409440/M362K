\documentclass[12pt]{article}

\usepackage{graphicx}
\usepackage[margin=1.0in]{geometry}
\usepackage{amsmath}
\usepackage{cases}
\usepackage{amsfonts}
\usepackage{amssymb}
\usepackage{grffile}
\usepackage{setspace}

%\setlength\parindent{0pt}

\author{Xiaohui Chen \\EID: xc2388}
\title{M 362K Synopses for 2/10}

\begin{document}
\maketitle
\begin{spacing}{2.0}

If a variable X is a numerically valued function whose domain is the sample space of probability experiment with a finite or countably finite number of outcomes, then we can say that X is a discrete random variable. A table which indicate the value of $Pr(X=x)$ is called the probability distribution. This table gives a concrete view of how the probability is distributed. Of course, each probability in the distribution should be larger than 0 and less of equal to 1 and the sum of all probabilities should be 1.

As for a cumulative distribution function, it follows $Pr(X=x_i)= F(x_i)-F(x_{i-1})$ and $F(\infty)=1$. The CDF function can be represented by ogive graph and CDF graph. 



\end{spacing}
\end{document}