\documentclass[12pt]{article}

\usepackage{graphicx}
\usepackage[margin=1.0in]{geometry}
\usepackage{amsmath}
\usepackage{cases}
\usepackage{amsfonts}
\usepackage{amssymb}
\usepackage{grffile}
\usepackage{setspace}

%\setlength\parindent{0pt}

\author{Xiaohui Chen \\EID: xc2388}
\title{M 362K Synopses for 1/29}

\begin{document}
\maketitle
\begin{spacing}{2.0}

The conditional probability that event A occurs given that event B occurred is $Pr(A|B)=\frac{Pr(A\cap B)}{Pr(B)}$. This means the probability of event A is affected when event B is observed. As a matter of fact, conditional probabilities can be represented in tree diagrams. For a path in a tree, the probability along that path is the conditional probabilities given that the prior events along the path have occurred. The probability of the outcome is the product of the numbers along the path.

One of the most important application of conditional probabilities is the Bayesian inference. In conditional probabilities, we know the probabilities of some effects given some causes. However, in reality we can only observe the effects and the causes are not obvious. Therefore we have to use conditional probabilities to calculate the probabilities of causes given the effects. This is called inference.

\end{spacing}
\end{document}