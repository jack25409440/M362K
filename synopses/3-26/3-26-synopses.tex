\documentclass[12pt]{article}

\usepackage{graphicx}
\usepackage[margin=1.0in]{geometry}
\usepackage{amsmath}
\usepackage{cases}
\usepackage{amsfonts}
\usepackage{amssymb}
\usepackage{grffile}
\usepackage{setspace}

%\setlength\parindent{0pt}

\author{Xiaohui Chen \\EID: xc2388}
\title{M 362K Synopses for 3/24}

\begin{document}
\maketitle
\begin{spacing}{2.0}

The expected value of $g(x)$ in a contunuous random variable is $E[g(x)]= \int_{-\infty}^{\infty} g(x)\cdot f(x) dx$. Similarly, the variance is also defined as $Var[X]= E[X^2]- E[X]^2$. The variance obeys the linear tansformation rule. Meanwhile, the mode of X is defined as the values such that $f(x)$ is at global maximum, where $f(x)$ is the probability density function. In order to calculate the percentile $100p^{th}$, we only have to let $p=Pr(X\le x_p)$ then calculate $x_p$. Therefore $x_{0.5}$ is defined as median. Finally, an alternative to calculate the expected value is $E[X]= A+ \int_{A}^{B} [1-F(x)] dx$. Here $(A,B)$ is the interval which the random variable lives on. 

\end{spacing}
\end{document}
