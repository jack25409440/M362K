\documentclass[12pt]{article}

\usepackage{graphicx}
\usepackage[margin=1.0in]{geometry}
\usepackage{amsmath}
\usepackage{cases}
\usepackage{amsfonts}
\usepackage{amssymb}
\usepackage{grffile}
\usepackage{setspace}

%\setlength\parindent{0pt}

\author{Xiaohui Chen \\EID: xc2388}
\title{M 362K Synopses for 3/10}

\begin{document}
\maketitle
\begin{spacing}{2.0}

The equation of Poisson distribution is $Pr(Z=k) = e^{-\lambda} \frac{\lambda^k}{k!}$. The expected value and the variance should both be $\lambda$. For a set of data, we have to verify whether the expected value(mean) and the variance are approximately equal. In this case, we can say that the dataset has the property of Poisson distribution.

Note that the expected values of two independent Poisson random variables can be sum up. This means $E[Z_1+Z_2]= \lambda_1 + \lambda_2$

\end{spacing}
\end{document}