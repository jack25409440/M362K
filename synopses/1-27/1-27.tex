\documentclass[12pt]{article}

\usepackage{graphicx}
\usepackage[margin=1.0in]{geometry}
\usepackage{amsmath}
\usepackage{cases}
\usepackage{amsfonts}
\usepackage{amssymb}
\usepackage{grffile}
\usepackage{setspace}

%\setlength\parindent{0pt}

\author{Xiaohui Chen \\EID: xc2388}
\title{M 362K Synopses for 1/27}

\begin{document}
\maketitle
\begin{spacing}{2.0}

The samples and events can be grouped as sets. A set is indeed a collection of objects. Each element in the set is an possible outcome. The universal set is indeed the sample space.

There are lots of set operations including union, intersection, complement, difference and subset. Also, we need to denote the cardinality of a set as $N(A)$, which is the number of elements in set A. Sets can be represented as Venn diagram, which clearly represents the relationship between sets and the universal set. In addition, any two sets, set A and set B, obey De Morgan's Law ($(A\cap B)' = A' \cup B'$ and $(A\cup B)'=A'\cap B'$).

There are three basic axioms of probability theory in the textbook. Also, we have to understand and utilize the negation rule and inclusion-exclusion rules. Those rules are fundamental in probability theory.

The Venn diagram can have a tabular form called Venn box diagram, which can clearly shows the relationship of the probabilities between two events A and B.
\end{spacing}
\end{document} 