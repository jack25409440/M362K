\documentclass[12pt]{article}

\usepackage{graphicx}
\usepackage[margin=1.0in]{geometry}
\usepackage{amsmath}
\usepackage{cases}
\usepackage{amsfonts}
\usepackage{amssymb}
\usepackage{grffile}
\usepackage{setspace}

%\setlength\parindent{0pt}

\author{Xiaohui Chen \\EID: xc2388}
\title{M 362K Synopses for 2/12}

\begin{document}
\maketitle
\begin{spacing}{2.0}

For a set of data, the word "average" can mean mode, median, midrange, or mean. The term "mean" means the expected value ($\mu_X= E[X]= \sum_i x_i*p(x_i)$). When calculating the mean od a transformed random variable, all we need to do is to replace $x_i$ with $f(x_i)$, which is the transformation of random variable.

When the number of data is odd, then the median is the middle term. Otherwise, it is the mean of the two middle terms. Midrange os halfway between the minimum and maximum value of the set of data. It is usually defined as $\frac{x_1+x_n}{2}$, where $x_1$ is the smallest value and $x_n$ is the largest value.

Mode can be defined as: (1) The value $x_i$ that occurs most frequently. (2) The two values that occurs most(with same frequency). (3) No mode otherwise.

In a set of data, a percentile is the percentage ranking of a specified data value. If the specified data value is not within the data set, then we can use linear interpolation to calculate the percentage. For random variables, we simply have to list all the random variables in sorted order in order to calculate the percentage.


\end{spacing}
\end{document} 