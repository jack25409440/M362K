\documentclass[12pt]{article}

\usepackage{graphicx}
\usepackage[margin=1.0in]{geometry}
\usepackage{amsmath}
\usepackage{cases}
\usepackage{amsfonts}
\usepackage{amssymb}
\usepackage{grffile}
\usepackage{setspace}

%\setlength\parindent{0pt}

\author{Xiaohui Chen \\EID: xc2388}
\title{M 362K Synopses for 3/31}

\begin{document}
\maketitle
\begin{spacing}{2.0}

Sometimes a random variable can be mixed. This mean sometimes it can be continuous and sometimes it can be discrete. In order to calcualte the expected value, we then have to split that to two cases and use the equations for calculating discrete and continuous random variables to figure out the expected value. 

One of the applications for mixed random variables is that insurance companies with dedutibles and/or caps can calcualte the expected premium a customer could get by calcuating the expected value of payments by the insurance company.

\end{spacing}
\end{document}
