\documentclass[12pt]{article}

\usepackage{graphicx}
\usepackage[margin=1.0in]{geometry}
\usepackage{amsmath}
\usepackage{cases}
\usepackage{amsfonts}
\usepackage{amssymb}
\usepackage{grffile}
\usepackage{setspace}

%\setlength\parindent{0pt}

\author{Xiaohui Chen \\EID: xc2388}
\title{M 362K Synopses for 2/26}

\begin{document}
\maketitle
\begin{spacing}{2.0}

In a discrete uniform distribution, the probability of each outcome is the same. A random variable X is said to have a discrete uniform distribution if its probability function is $Pr(X=x)=p(x)=\frac{1}{n}$ for $x=1,2,\ldots, n$. Therefore, $E[X]=\frac{n+1}{2}$ and $Var[X]= \frac{n^2-1}{12}$. 

A Bernoulli trial is an experiment that has two outcomes. When a random variable X is a Bernoulli random variable and $X=1$, then it is considered success. If $X=0$, then it is considered a failure. In this case, $E[X]=p$ and $Var[X]=p*q=p*(1-p)$. For a trial of n times, $E[X]=n*p$ and $Var[X]=n*p*q=n*p*(1-p)$. 

\end{spacing}
\end{document} 