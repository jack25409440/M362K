\documentclass[12pt]{article}

\usepackage{graphicx}
\usepackage[margin=1.0in]{geometry}
\usepackage{amsmath}
\usepackage{cases}
\usepackage{amsfonts}
\usepackage{amssymb}
\usepackage{grffile}
\usepackage{setspace}

%\setlength\parindent{0pt}

\author{Xiaohui Chen \\EID: xc2388}
\title{M 362K Synopses for 3/3}

\begin{document}
\maketitle
\begin{spacing}{2.0}

The geometric distribution has the probability function $Pr(X=k)= p*(1-p)^k$. X is called geometric random variable with parameter p. The expected value is $E[X]= \frac{1-p}{p}$ and the variance is $Var[X]= \frac{1-p}{p^2}$

A negative binomial process means

(a) The trials are identical

(b) Each trial is independent

(c) The random variable M denotes the number of failures prior to the $r^{th}$ success

(d) The probability of success is $p$ and the probability of failure is $q=1-p$

Such distribution is given by $Pr(M=k)={}_{r+k-1}C_{k} p^r*(1-p)^k$. The expected value of failure is $E[M]= r*\frac{1-p}{p}$ and the variance is $Var[M]= r*\frac{1-p}{p^2}$

The hyper-geometric distribution is given by $Pr(X=k)= \frac{{}_{G}C_{k}*{}_{B}C_{n-k}}{{}_{G+B}C_{n}}$ where X is a hyper-geometric random variable. The expected value is given by $E[X]= n*\left( \frac{B}{B+G} \right)$ and the variance is given by $Var[X]= n*\left( \frac{B}{B+G} \right)*\left( \frac{G}{B+G} \right)*\left( \frac{B+G-n}{B+G-1} \right)$.

\end{spacing}
\end{document} 