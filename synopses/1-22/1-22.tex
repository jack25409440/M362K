\documentclass[12pt]{article}

\usepackage{graphicx}
\usepackage[margin=1.0in]{geometry}
\usepackage{amsmath}
\usepackage{cases}
\usepackage{amsfonts}
\usepackage{amssymb}
\usepackage{grffile}
\usepackage{setspace}

\setlength\parindent{0pt}

\author{Xiaohui Chen \\EID: xc2388}
\title{M 362K Synopses Work for 1/22}


\begin{document}
\maketitle
\begin{spacing}{2.0}

Combinational and permutational problems can be interpreted by sampling and distributing. \textbf{Sampling} means choosing subsets from a set of n distinguishable objects while \textbf{distribution} means assigning markers to the n objects.

While solving sampling problems, we have to consider whether the samples are with replacement and whether the order matters. Therefore we have to consider four cases.

When we consider distribution problems, we have to take into account whether the balls are distinguishable and whether the urns are exclusive. The most complicated case is dealing with samples with replacement when order does not matter. For a sample of such r objects from n distinguishable ones, the total number of samplings is ${}_{n+r-1}C_{r}=\left(
\begin{array}{c}
n+r-1 \\
r
\end{array}
\right)$. In fact, section 1.3.3 in textbook provides a detailed graph of sampling and distribution problems.

Of course, sampling and distribution have lots of applications. The textbook mentions the binomial and multinomial theorems, poker hands and the Powerball lottery. There are many other applications for us to explore.


\end{spacing}
\end{document}